\documentclass[compress, english, aspectratio=169]{beamer}

\usepackage{pythontex} % To run Python code
\usepackage[utf8]{inputenc}
\usepackage{amssymb}
%\usepackage{graphicx}
\usepackage{csquotes}
\graphicspath{{images/}}

\usepackage[scheme=plain]{ctex} % To display Chinese characters

\usetheme[navigation]{UMONS}
%\usetheme[navigation, no-subsection, no-totalframenumber]{UMONS}

\newcommand{\IR}{\mathbb{R}}

%% Presentation information
\title{Industrial Instrumentation: Design of Experiments}
\subtitle{GUI project with Python and PyQt5}
\author[V. \textsc{Stragier}]{Vincent \textsc{Stragier}}
\date{16th of June}
\institute[| FPMs]{%
  Faculty of Engineering\\
  University of  Mons
  \\[2ex]
  \includegraphics[height=6ex]{LOGO_UMONS_EN}\hspace{2em}
  \includegraphics[height=6ex]{LOGO_FPMs}
}

\begin{document}
% Setup pythontex to be able to import another module located in other path
\begin{pyconcode}
import os;
os.chdir('../')
import sys
sys.path.append("..\\src")
sys.path.append(os.getcwd())
sys.path = list(set(sys.path))
\end{pyconcode}

%% Title
\begin{frame}[plain]
  \titlepage
\end{frame}

%% Table of content
\begin{frame}
  \tableofcontents
\end{frame}

%% Proverb
\begin{frame}
\frametitle{Chinese proverb}
\begin{block}{读万卷书不如行万里路。(Dú wànjuànshū bù rú xíng wànlǐlù)}
It's better to walk thousands of miles than to read thousands of books.
\end{block}
\end{frame}

%% Theory
\section{Theory about the 2 factorial Design of Experiments}
\subsection{Factorial design}
\begin{frame}
\frametitle{Factorial design}
\begin{block}{Goals}
\begin{itemize}
\item efficient measurement:
\begin{itemize}
\item quickly arrive at the best possible results
\item omit unnecessary trials
\item give results with the lowest uncertainty
\item progress without failure
\item establish a model for the studied phenomenon
\item discover the optimal solution
\end{itemize}
\end{itemize}
\end{block}

\begin{equation}
m = m_0^{n}
\end{equation}

where $m$ is the number of measurements, $m_0$ is the number of levels and $n$ is the number of factors that you want to adjust. For the program $m_0=2$ is assumed. It will define the number of measurements to take in the experimental space.
\end{frame}

\section{Python implementation}
\subsection{How to program}
\begin{frame}
\frametitle{How to program}
\begin{block}{Setup}
\begin{itemize}
\item sheets of paper and pen
\item book(s) or internet connection
\item a computer
\item an IDE
\end{itemize}
\end{block}
\end{frame}

\begin{frame}
\frametitle{How to program}
\begin{block}{Methodology}
\begin{itemize}
\item up-to-date state-of-the-art
\item scheme of your program
\item use the classical naming standards
\item testing and debugging method
\item document your code
\item implement the minimal functions first
\item do your research before asking anything
\end{itemize}
\end{block}
\end{frame}



\subsection{Core functions}
\subsubsection{Algorithm}
\begin{frame}
\frametitle{Core functions}
\framesubtitle{Algorithm}
\begin{align*}
\hat{\underline{a}} = \left(X^T X\right)^{-1} X^T \underline{y}
\end{align*}

\noindent where: 

\begin{description}
\item[$\hat{\underline{a}}$] is a vector of coefficients (of shape $p\times 1$)
\item[$\underline{y}$] is a vector of measures (of shape $p\times 1$)
\item[$X^T$] is the transposed form of the matrix $T$
\item[$\left(X^T X\right)$] is the square information matrix (of shape $p\times p$)
\item[$\left(X^T X\right)^{-1}$] is the square dispersion matrix (of shape $p\times p$)
\end{description}
\end{frame}

\subsubsection{Aim}
\begin{frame}
\frametitle{Core functions}
\framesubtitle{Aim}
The aim is to compute the estimation of the coefficients to provide a model for the experiment. In our case the following model can be used:

\begin{align*}
y = \hat{a}_0 + \sum^n_{i=1} \hat{a}_i x_i + \sum^n_{i=1} \sum^n_{j=1} \hat{a}_{ij} x_i {x}_j + ...
\end{align*}

To ease the resolution of the coefficients, $\hat{X}$ is defined as followed:
\begin{equation*}
\hat{X} = \left(X^T X\right)^{-1} X^T
\end{equation*}
\end{frame}

\subsubsection{Console full example}
\begin{frame}[fragile]
\frametitle{Core functions}
\framesubtitle{Console full example}
\begin{pyconsole}
import design_of_experiments as doe
X_hat = doe.gen_X_hat(n=3)
X_hat
\end{pyconsole}
\end{frame}

\begin{frame}[fragile]
\frametitle{Core functions}
\framesubtitle{Console full example (continued)}
\begin{pyconsole}
# Speed [mg/min]
y1 = [53, 122, 20, 125, 48, 70, 68, 134]
# Cobalt Content [ppm]
y2 = [4100, 3510, 3950, 1270, 4870, 2810, 7750, 3580]
# Compute the coefficients
import numpy as np
a_hat_1 = np.dot(X_hat, y1)
a_hat_2 = np.dot(X_hat, y2)
a_hat_1
a_hat_2
\end{pyconsole}
\end{frame}

\subsection{GUI programming (baselines)}
\begin{frame}
\frametitle{GUI programming (baselines)}
In the frame of this project, I have used the book \textit{Understanding Optics with Python, Vasudevan Lakshminarayanan, Hassen Ghalila, Ahmed Ammar, and L. Srinivasa Varadharajan,  CRC Press, 2018}.
\begin{block}{Main steps}
\begin{itemize}
\item design the GUI on the Qt Designer tool
\item convert the .ui file generated in a Python script
\item create a new script that will import the generated script and make the links between the main script and the GUI script
\end{itemize}
\end{block}
Useful resource: \url{www.learnpyqt.com/}
\end{frame}

\section{Demonstration}
\subsection{Requirements}
\begin{frame}
\frametitle{Requirements}
\begin{block}{Program and modules}
\begin{itemize}
\item Python 3.8.0 or above
\item the source code (\url{https://github.com/2010019970909/design_of_experiments})
\item any working LaTeX installation
\item Python modules:\\
\begin{description}
\item[scikit] for the statistics
\item[numpy] for the math (matrix dot product, etc.)
\item[matplotlib] for the plots
\item[PyQt5] for the GUI
\item[PyQt5-tools] to further development
\end{description}
\end{itemize}
\end{block}
\pyb{py -3.8 -m pip install --upgrade sklearn numpy matplotlib PyQt5 PyQt5-tools}
\end{frame}

\subsection{Live demo or video}
\begin{frame}
\frametitle{Live demo or video}
\begin{block}{Video}
\url{https://youtu.be/2I9Mx7gjNnI}
\end{block}
\end{frame}

\section{Conclusion}
\begin{frame}
\frametitle{Conclusion}
\begin{block}{Features}
\begin{itemize}
\item minimal module for DoE
\item GUI:
\begin{itemize}
\item input up to $2^8$ measures
\item plot charts (coefficients, Pareto, Henry)
\item save the result in a file (.csv or .xls(x))
\item open a file to compute the results
\item evaluation of mathematical expression in the input
\end{itemize}
\end{itemize}
\end{block}
\end{frame}

\begin{frame}
\frametitle{Conclusion (continued)}
\begin{block}{Improvements}
\begin{itemize}
\item add an interpolation engine (knowing the limits of the experimental space)
\item add the uncertainties on each coefficient and in the charts
\item add the possibility of studying multiple quantities at once
\item add more charts to enhance the coefficients analyse (contour plot, 2 by 2 correlation)
\item create a better import tool for the files
\item improve the design of the GUI
\item improve the runtime if possible
\end{itemize}
\end{block}
\end{frame}
\end{document}
